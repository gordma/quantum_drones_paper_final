
\section{Introduction}

Recent large-scale pilot programs by UPS Flight Forward, Zipline, and Wing vividly 
demonstrate the practical feasibility and efficiency of drone-based logistics, delivering 
small parcels rapidly while significantly reducing labor and carbon emissions 
\cite{murray2020survey}. However, scaling these drone delivery systems from controlled 
pilots to widespread deployment introduces substantial operational complexity. Real-world 
drone operations must simultaneously respect stringent battery energy constraints, payload 
limitations, no-fly corridors, and practical considerations such as ``wave-based'' launch 
schedules to accommodate repeated drone flights and realistic operational constraints. 
While each of these constraints can, in principle, be embedded into exact optimization 
frameworks like mixed-integer linear programming (MILP) or branch-and-price formulations, 
industry experience consistently indicates that solver runtime is rarely the critical 
bottleneck. Instead, the dominant cost in real-world deployments arises from extensive 
human modeling effort and expertise required to define, tune, and refine these optimization 
models \cite{kallrath_mixed_2000, rossi2006hcp}. Consequently, logistics organizations 
frequently default to heuristic methods, sacrificing potential optimality to sidestep the 
prohibitively high modeling overhead.

Combinational optimization has simultaneously emerged as a leading candidate for near-term 
quantum computing applications. Yet early quantum implementations of drone-routing 
problems---particularly those relying on Quadratic Unconstrained Binary Optimization 
(QUBO) encodings---highlight a severe practical challenge: qubit resource explosion. Even 
minimal drone-routing scenarios quickly surpass the capabilities of current Noisy 
Intermediate-Scale Quantum (NISQ) devices; for instance, encoding a modest four-node, 
six-edge scenario as a QUBO problem already requires hundreds of binary variables after 
MILP linearization, far exceeding practical quantum budgets \cite{davies_quantum_2024}. 
To address this qubit explosion, recent research has introduced ``qubit-efficient'' 
encoding schemes, achieving exponential qubit savings by storing classical variables 
logarithmically in quantum registers \cite{leonidas2024qubovrp, tan_qubit-efficient_2021, 
sundar_qubit-efficient_2024}. However, these schemes generally retain critical limitations 
of traditional QUBO methods, such as heuristic approximation methods that introduce 
integrality gaps, reliance on non-convex surrogate objectives, or significantly deeper 
quantum circuits to retrieve necessary correlations.

These persistent drawbacks have driven recent literature to call explicitly for alternative 
optimization paradigms---approaches that retain the significant qubit compression advantages 
of logarithmic encodings while entirely sidestepping the traditional QUBO formulation 
\cite{chang2025, gabbassov2023, fuchs2022}. Instead, quantum optimization is increasingly 
viewed as a probabilistic and risk-aware sampler integrated within hybrid quantum-classical 
pipelines \cite{matsuyama2025, albarran2024}. Such methods explicitly leverage classical 
preprocessing to manage feasibility constraints, allowing quantum devices to specialize 
purely in combinational subset selection without incurring the steep complexity penalties 
common to QUBO encodings.

In this paper, we introduce precisely such a hybrid quantum-classical approach tailored 
explicitly for drone routing. Our pipeline first leverages efficient classical 
preprocessing---via a greedy soft-max generator---to produce a large pool of feasible 
drone routes, explicitly satisfying battery, payload, and flight-path constraints. The 
quantum algorithm then exclusively focuses on selecting an optimal subset of these feasible 
route-wave pairs, encoded compactly using only 13 qubits via logarithmic indexing. 
Employing a Conditional-Value-at-Risk (CVaR) quantum optimizer 
\cite{barkoutsos_improving_2020} integrated with Bayesian hyperparameter optimization 
(via Optuna's Tree-Parzen Estimator), we rigorously demonstrate a 6--8\% reduction in 
average routing cost compared to a well-known Genetic Algorithm baseline. While the 
Genetic Algorithm may not represent a near-optimal, publication-grade baseline, it remains 
a strong, reliable black-box solver widely adopted in industry scenarios where 
custom-crafted optimization solutions are impractical or prohibitively costly.

Our main contributions thus include:
\begin{enumerate}
\item Introducing the first fully risk-aware quantum routing optimizer, achieving superior 
results over established classical heuristics while utilizing minimal qubit resources.
\item Demonstrating practical quantum advantage through a hybrid approach that shifts 
feasibility logic entirely to classical preprocessing, thus drastically reducing quantum 
hardware complexity.
\item Providing an open-source, straightforward, and extendable toolkit, enabling immediate 
community adoption and further exploration into hybrid quantum-classical logistics 
optimization.
\end{enumerate}

The paper proceeds as follows. In Section~\ref{sec:related_work}, we critically survey 
relevant literature, detailing the shift away from QUBO toward alternative optimization 
paradigms. Section~\ref{sec:problem} formally defines the Subset-Route Drone VRP problem 
and its associated cost oracle. Our quantum methods, including qubit encoding, CVaR-VQE 
algorithm, and Bayesian optimization, are detailed in Section~\ref{sec:methods}. Extensive 
experimental validation and comparisons with classical baselines appear in 
Section~\ref{sec:experiments_results}, followed by conclusions, limitations, and future 
research directions in Section~\ref{sec:discussion}.
