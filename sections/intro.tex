% \section{Introduction}

% \subsection{Drone logistics at an engineering inflection point}

% Recent large-scale pilot programmes by UPS\,Flight Forward, Zipline, Wing and others demonstrate that unmanned aerial vehicles (UAVs) can deliver small parcels in minutes while lowering both labour and \text{CO\textsubscript{2}} footprints.\cite{murray2020survey}  
% Yet each real-world deployment must respect \emph{simultaneous} constraints: limited battery energy, strict payload caps, no-fly corridors, and air-traffic ``wave’’ launch schedules mandated by regulators.  
% In principle every such rule can be embedded in a mixed-integer programming (MILP) or branch-and-price formulation, but industrial case studies consistently report that \emph{human modelling effort---not CPU time---dominates project cost}.  
% Bangert calls data cleansing and modelling ``the steps that require \emph{most} of the human time and effort’’ in optimisation projects\cite{kallrath_mixed_2000}, while hybrid CP/column-generation surveys list bespoke pricing routines as a \emph{major barrier}\cite{rossi2006hcp}.  
% Even powerful commercial MILP solvers such as Gurobi or CPLEX only shine after weeks of expert model sculpting and penalty tuning.
% Consequently, organisations favour fast heuristic tooling even when exact methods could yield better objective values.

% Simultaneously, combinatorial optimisation is seen as one of the most promising application domains for near-term quantum computing.
% Early quantum drone-routing studies, such as Davies and Kalidindi~\cite{davies_quantum_2024}, have shown that even minimal QUBO encodings remain prohibitively large for NISQ-era hardware: a four-node, six-edge drone-routing scenario already consumes 275 binary variables after MILP linearisation, far exceeding realistic quantum budgets.
% Recent “qubit-efficient” encoding schemes, notably by Leonidas et al.\cite{leonidas2024qubovrp} and related studies\cite{tan_qubit-efficient_2021,sundar_qubit-efficient_2024}, achieve exponential compression by representing classical variables as indices stored logarithmically in quantum registers, thereby significantly shrinking the qubit count.
% However, such minimal encodings often come at a steep practical cost: either heuristic approximation methods (e.g., minimal Bloch encodings with mean-field Sherali–Adams integrality gaps~\cite{tan2021qaoa_minimal}), non-convex surrogate objectives (Pauli-Correlation Encoding~\cite{sciorilli2023pce}), or increased circuit depths to retrieve correlations (Qubit-efficient QAOA~\cite{sundar2023qubit_efficient}).
% Our log-index approach takes a fundamentally simpler route: we retain the logarithmic index compression but completely avoid direct QUBO encoding and non-convex surrogates by shifting feasibility and constraint-handling entirely to an efficient classical preprocessing stage. With just $\mathcal{O}(log_2 N)$-qubits we are able to represent $N$ candidate routes, allowing the variational quantum loop to focus solely on solving the combinatorial subset selection—without the depth, complexity, or integrality penalties common to existing compression schemes.

% \subsection{From feasibility to combinatorial choice: the Subset-Route D-VRP}

% We adopt a middle-ground abstraction tailored to NISQ hardware.  
% First, a greedy soft-max generator creates a pool of $N=2\,048$ \emph{feasible} routes on a 30--40-node tactical graph, fully respecting battery, payload and no-fly constraints.  
% The operational decision therefore collapses to selecting exactly $k = 10$ route--wave pairs\footnote{Four launch waves $\Rightarrow \lceil\log_2 4\rceil = 2$ qubits for wave index.} out of this pool.  
% The search space contains ${2\,048 \choose 10}\times4^{10}\!\approx 2.6\times10^{26}$ possibilities---far beyond exhaustive enumeration but representable with only
% %
% \[
% \lceil\log_2 N\rceil + \lceil\log_2 4\rceil = 11 + 2 = 13
% \]
% %
% qubits via index encoding.  All feasibility reasoning remains in classical code; quantum resources are reserved for the \emph{combinatorial} challenge.

% \subsection{Why classical baselines plateau}

% A well-tuned Genetic Algorithm (GA) is the de-facto black-box baseline: it requires only the same cost oracle we already possess for feasibility checking.  
% However, empirical GA improvement typically stalls after $\mathcal{O}(10^{5})$ oracle calls on noisy surfaces, and the algorithm is \emph{risk-neutral}: it optimises the mean of a heteroscedastic objective without regard to tail costs.

% \subsection{Quantum as a stochastic search engine: gap in the literature}

% A growing body of work uses quantum hardware to \emph{sample} from rich distributions---reservoir computing, qGANs, probabilistic QAOA---rather than to collapse onto a single ``winner’’ bit-string.\cite{schuld2019quantum}  
% Yet existing quantum VRP studies still target toy instances ($\le 50$ nodes), ignore battery/wave constraints, and consume 100--500 qubits for direct QUBO encodings.\cite{vogel2022quantumvrp}  
% Whether NISQ devices can \emph{outperform strong classical baselines on realistic, constraint-heavy drone routing} therefore remains open.

% \subsection{Our compressed, risk-aware pipeline}

% We answer that question with a systematic approach:

% \begin{enumerate}[leftmargin=*,nosep]
% \item \textbf{Compressed log-index encoding} reduces the decision register to 13 qubits for the full 2\,048\,$\times$\,4 design space.
% \item \textbf{Millisecond black-box oracle} evaluates any candidate set in $\approx1$\,ms, inheriting exactly the same engineering cost as the GA baseline.
% \item \textbf{CVaR objective ($\alpha\!\approx\!0.05$)} focuses optimisation on the worst-cost tail, naturally down-weighting bit-strings that appear with vanishing shot probability.
%       Heteroscedastic noise stems from (i) quantum shots (binomial variance $p(1-p)$) and (ii) Monte-Carlo resampling of $2\,000$ route subsets.
%       CVaR truncates both noise sources.
% \item \textbf{Optuna TPE Bayesian search} exploits the smoother, risk-focused landscape without assuming homoscedastic Gaussian noise, yielding a synergy we do not observe with na\"ive gradient-free methods.
% \end{enumerate}

% \subsection{Headline results}

% On the 30--40-node instance our $1$--$2$-layer CVaR-VQE achieves a \textbf{6--8\,\% lower mean cost} than a mutation-swept GA under an identical budget of $10^{5}$ oracle evaluations, and the edge persists under a calibrated \texttt{ibm\_fez} noise model.  
% The entire pipeline is released under the MIT licence and comprises $<\!1\,000$ lines of Python.

% \subsection{Contributions}

% \begin{enumerate}[leftmargin=*,nosep]
% \item First risk-aware quantum optimiser to surpass a tuned GA on a realistic battery- and wave-constrained D-VRP with only 13 qubits.
% \item Demonstrates how shifting feasibility to a classical generator plus log-index encoding slashes qubit requirements without manual MILP modelling.
% \item Provides an open, engineering-light toolkit that invites hybrid exact extensions (e.g.\ column generation) for future work.
% \end{enumerate}

% \subsection{Roadmap}

% Section~\ref{sec:problem} formalises the Subset-Route D-VRP and the black-box oracle.  
% Section~\ref{sec:methods} details the compressed-qubit VQE, CVaR objective and Bayesian optimiser.  
% Section~\ref{sec:experiments} summarises experimental settings; Section~\ref{sec:results} presents results and ablation studies.  
% We conclude with limitations and future directions in Section~\ref{sec:discussion}.


\section{Introduction}

Recent large-scale drone delivery pilots by UPS Flight Forward, Zipline, and Wing have shown the potential of unmanned aerial vehicles (UAVs) to deliver small parcels rapidly, reducing both labour and $CO_2$ footprints \cite{murray2020survey}. Yet deploying drones at scale requires satisfying multiple simultaneous operational constraints: limited battery capacity, payload restrictions, no-fly corridors, and mandated "wave-based" launch schedules. While in principle each constraint can be encoded as a Mixed-Integer Linear Program (MILP) or via branch-and-price frameworks, industry experience consistently shows that the greatest barrier is not solver performance, but rather the significant \emph{human modelling effort and expertise} needed to formulate and tune these models \cite{bangert2012industrial,rossi2006hcp}. As a result, many logistics organisations default to simpler heuristic methods, accepting suboptimal solutions to sidestep the intensive modelling overhead associated with exact optimisation techniques.

Simultaneously, combinational optimisation problems have emerged as a leading application domain for near-term quantum computing. Yet early quantum approaches to drone mission planning highlight a persistent challenge: even minimal quadratic unconstrained binary optimisation (QUBO) encodings quickly become prohibitively large for current quantum hardware. For example, Davies and Kalidindi \cite{davies_quantum_2024} demonstrate that encoding a drone-routing scenario with just 4 nodes and 6 edges already requires 275 binary variables post-MILP linearisation—far beyond the quantum budget available in the Noisy Intermediate-Scale Quantum (NISQ) era. To mitigate this qubit explosion, recent "qubit-efficient" encoding schemes have been introduced, notably by Leonidas et al.\cite{leonidas2024qubovrp}, Tan et al.\cite{tan_qubit-efficient_2021}, and Sundar and Dupont \cite{sundar_qubit-efficient_2024}. These schemes achieve exponential compression by storing classical variables logarithmically in quantum registers. However, they introduce substantial drawbacks: heuristic approximation (Bloch-encoded mean-field relaxations), non-convex surrogate objectives (Pauli-Correlation Encoding), or deeper quantum circuits to capture higher-order correlations (Qubit-efficient QAOA).

In contrast, our work introduces a novel quantum-compatible pipeline specifically designed to avoid both the combinational explosion of traditional QUBO encodings and the heuristic complexity of recent minimal-qubit approaches. We achieve this by decomposing the drone routing problem (D-VRP) into two complementary steps: first, using a classical greedy soft-max route generator to construct a pool of $N=2,048$ feasible routes on a realistic 30–40 node tactical graph, ensuring all battery, payload, and must-fly route constraints are explicitly satisfied classically. The quantum optimisation then becomes purely combinational:
selecting exactly \(k = 10\) missions from this fixed pool and assigning
each to one of \(M = 4\) possible launch waves—allowing each drone to fly
multiple missions, spread across different waves.
This yields a combinational search space of 
${2,048 \choose 10} \times 4^{10} \;\approx\; 3.6 \times 10^{32}$,
far beyond exhaustive enumeration but fully representable using a
compact 13-qubit register.
Crucially, our logarithmic indexing scheme requires only
\(13\) qubits
\((\lceil\log_2 N\rceil + \lceil\log_2 M\rceil = 11 + 2 = 13)\),
achieving multiple orders-of-magnitude savings over traditional QUBO
encodings—while still consistently outperforming strong classical
black-box optimisers such as Genetic Algorithms.


We combine this minimal qubit representation with a millisecond-scale classical oracle, a Conditional-Value-at-Risk (CVaR)\cite{barkoutsos_improving_2020} objective targeting worst-case outcomes , and a Bayesian hyper-parameter optimisation framework (Optuna's Tree-Parzen Estimator, TPE) designed explicitly for noisy, heteroscedastic cost surfaces. On realistic scenarios involving drone wave scheduling, we demonstrate that our shallow-depth (1–2 layer) CVaR-VQE achieves a \textbf{6–8 \% lower mean cost} compared to a rigorously tuned Genetic Algorithm (GA) baseline under identical computational budgets of $10^5$ evaluations. Notably, this quantum advantage persists even when incorporating realistic hardware noise calibrated to IBM's \texttt{ibm\_fez} quantum backend. Our approach thus offers a practical and scalable quantum route-planning solution achievable with current-generation NISQ devices, and we release our open-source pipeline under an MIT licence to enable immediate community adoption.

Our main contributions are:
\begin{enumerate}
\item Presenting the first risk-aware, qubit-compressed quantum optimiser that robustly surpasses a carefully tuned GA baseline on a realistic D-VRP scenario using only 13 qubits.
\item Demonstrating how to achieve practical combinational quantum advantage by shifting feasibility logic entirely to a classical route-generator and using minimal, logarithmic indexing without extensive manual QUBO modelling.
\item Providing an open, easy-to-use toolkit that facilitates future research into hybrid exact-quantum methods (e.g., column generation), larger problem instances, and realistic hardware deployment.
\end{enumerate}

The paper is structured as follows: Section \ref{sec:methods} details our problem formalisation, compressed-qubit encoding, CVaR-VQE algorithm, and Bayesian optimisation. Section \ref{sec:experiments} describes experimental setups and benchmarks. Results and extensive validation studies appear in Section \ref{sec:results}, followed by conclusions, limitations, and outlooks on future work in Section \ref{sec:discussion}.
