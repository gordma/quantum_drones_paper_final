\section{Introduction}

Same-day delivery has evolved from a niche premium into a widespread expectation.
In urban logistics, providers now face pressures to achieve sub-hour deliveries,
dramatically escalating operational complexity.
This urgency intensifies last-mile logistics—which already constitute more than 50\%
of total fulfillment costs—and clashes with increasing demands for carbon efficiency,
sustainability, and regulatory compliance.
Autonomous drone fleets represent a compelling solution to these challenges,
with pilot programs by UPS Flight Forward, Zipline, and Wing demonstrating their
potential to deliver small parcels rapidly while significantly reducing labor and
carbon emissions \cite{murray_multiple_2020}.
Despite this promise, widespread commercialization remains elusive, largely due to
the complex optimization challenges inherent in scaling from controlled pilots to
real-world deployment.

Practical drone logistics at scale involve combinatorial optimization problems with enormous
search spaces and complex, intertwined constraints such as battery limits, no-fly zones,
wave-based launch schedules, and tight customer time windows.
While exact optimization methods such as Mixed-Integer Linear Programming (MILP) or Branch-and-Price
can theoretically solve these problems, industry experience consistently identifies the dominant bottleneck
as the extensive human effort required to model and refine these optimization scenarios,
rather than solver runtime \cite{hildebrandt_implementation_1981,boschetti_contemporary_2024,kallrath_mixed_2000}.
This modeling overhead—requiring specialized expertise to define, tune, and refine optimization
models—often proves prohibitively expensive in real-world deployments.
Consequently, drone logistics providers frequently resort to heuristic methods,
sacrificing optimality to circumvent this modeling burden.

Quantum computing has simultaneously emerged as a potential remedy,
offering a different paradigmatic approach for addressing large-scale combinatorial optimization problems.
Yet initial quantum implementations of drone-routing and similar Vehicle Routing Problems (VRP)
using standard Quadratic Unconstrained Binary Optimization (QUBO) encodings quickly encounter severe limitations:
a modest four-node scenario already demands hundreds of binary variables after MILP linearization, despite the efforts to reduce variable-count,
exceeding current quantum hardware capabilities \cite{davies_quantum_2024}. 

To address this qubit explosion, a diverse family of \emph{qubit-efficient encoding schemes} has recently emerged.
Minimal encodings introduced by Tan et al.~\cite{tan_qubit-efficient_2021}, 
and subsequently refined by Huber et al. for a domain-specific transaction settlement problem ~\cite{huber_exponential_2024} and Leonidas et al. for vehicle routing problem with time windows for shipping logistics ~\cite{leonidas_qubit_2024}, 
compress resources exponentially by encoding each variable within the Bloch sphere and extracting correlations through register projections.
However, these methods inherently remain a mean-field linear-relaxation heuristic,
which can incur worst-case integrality gaps of at least 2x on canonical combinatorial problems such as Max-Cut~\cite{charikar_integrality_2009}.
Pauli-Correlation Encoding (PCE), proposed by Sciorilli et al.~\cite{sciorilli_towards_2025}, 
captures higher-order Pauli correlations but optimizes a non-convex \(\tanh\) surrogate objective, 
without clear connections to established LP or SDP relaxations.
Meanwhile, the qubit-efficient Quantum Approximate Optimization Algorithm (QAOA) by Sundar and Dupont~\cite{sundar_qubit-efficient_2024} 
maps \(N\) classical bits onto \(q = d + \log_2(N/d)\) qubits, demonstrating parameter clustering on spin-glass instances, 
but remains limited by sampling from depth-\(p\) circuits.
In essence, these advanced encoding methods typically seek single bit-string solutions, 
trading circuit depth and non-convexity challenges for reduced qubit count, yet still fail to fully escape
limitations in practical scaling and constraint management.

An alternative, complementary approach gaining traction is the integration of quantum components as
probabilistic, risk-aware samplers embedded within sophisticated classical preprocessing workflows.
Recent advances frame the quantum device as part of a hybrid quantum-classical optimization loop,
explicitly shifting feasibility management to classical preprocessing while allowing quantum
hardware to specialize purely on combinatorial subset selection \cite{matsuyama_sampling-based_2025}.
Such hybrid pipelines naturally incorporate modern AutoML techniques like Bayesian optimization (Optuna),
which systematically replace manual parameter tuning, 
making them especially appealing for practical industry deployment \cite{akiba_optuna_2019,tibaldi_bayesian_2023,caramanis_optimizing_2023}.

\textbf{Our approach and contributions.} 
In this paper, we introduce a novel hybrid quantum-classical optimization framework specifically tailored for drone logistics,
that leverages advanced classical preprocessing to rigorously manage feasibility constraints,
leaving quantum hardware to specialize purely on subset selection using minimal qubit resources.
Our approach first employs a greedy soft-max heuristic to generate a manageable pool of feasible drone routes,
which are then encoded compactly using just 13 qubits via logarithmic indexing.
By integrating Bayesian hyperparameter optimization (Optuna's Tree-Parzen Estimator) 
with a Conditional-Value-at-Risk (CVaR)-optimized Variational Quantum Eigensolver (VQE),
we demonstrate a robust 6–8\% cost reduction over a Genetic Algorithm baseline—a strong,
reliable solver widely adopted in industry scenarios where custom-crafted optimization
solutions are impractical or prohibitively costly.

The key contributions of this work are as follows:
\begin{enumerate}
\item The first quantum-classical logistics pipeline explicitly designed to minimize both human modeling overhead and qubit count simultaneously.
\item Demonstrated quantum advantage in a real-world-inspired logistics problem, using a CVaR optimization strategy and minimal encoding to achieve superior results over established industry-relevant heuristics.
\item A comprehensive, reproducible open-source toolkit facilitating immediate community adoption and further exploration of hybrid quantum-classical optimization methods in logistics.
\end{enumerate}

The remainder of the paper is structured as follows. 
In Section~\ref{sec:related_work}, we situate our contributions within existing approaches and recent developments in quantum optimization methods.
Section~\ref{sec:problem} rigorously defines the Subset-Route Drone VRP, providing explicit mathematical and operational constraints.
In Section~\ref{sec:methods}, we detail the classical preprocessing, qubit-efficient encoding, and quantum optimization procedures.
Extensive experimental results, benchmarking against classical methods, and detailed discussion appear in Section~\ref{sec:experiments},
followed by concluding remarks, practical considerations, and avenues for future research in Section~\ref{sec:discussion}.