\section{Results}
\label{sec:results}

We organize our findings around five core questions that assess the viability 
of quantum-assisted drone routing. Our experimental design builds on comprehensive 
parameter sweeps and ablation studies (detailed in Appendix~\ref{sec:appendix_parameters}) 
that identified optimal configurations for both quantum and classical methods.

\subsection{Quantum vs Classical Performance Comparison}

The 13-qubit quantum solver achieves a 7.6\% cost reduction with statistical 
significance over the tuned GA baseline.

Figure~\ref{fig:learning_curves} shows optimization progress under an equal
200{,}000-oracle budget. CVaR-VQE achieves substantial cost reduction over 
our rigorously tuned GA baseline (mutation rate optimized across extensive 
hyperparameter sweep, Appendix~\ref{sec:appendix_ga_tuning}). A one-sided Welch test confirms statistical significance 
($p<0.001$), establishing a robust quantum advantage despite the compressed 
13-qubit encoding.

\begin{figure}[h]
  \centering
  \includegraphics[width=.7\linewidth]{fig/learning_curves_comparison.pdf}
  \caption{Convergence trajectories on the 200{,}000-oracle budget. 
           Shaded bands show $\pm1$ standard deviation over 30 independent trials.}
  \label{fig:learning_curves}
\end{figure}

The quantum method also demonstrates superior consistency, with lower 
standard deviation than the GA baseline, indicating reliable 
performance across random seeds.

\subsection{Optimizer Sensitivity Analysis}

Aggressive TPE significantly outperforms Default TPE, confirming that CVaR 
benefits from exploitation-heavy search. Comprehensive optimizer ablation studies 
(Appendix~\ref{sec:appendix_optimizer_ablation}) demonstrate that quantum advantage 
persists across all Bayesian optimization strategies, ruling out hyperparameter 
optimization as the source of performance gains.

Table~\ref{tab:main_results} compares all methods under the fair 
200{,}000-evaluation budget, with the GA baseline representing the best performance 
achievable after extensive mutation rate optimization. The quantum approach demonstrates robust 
superiority over this rigorously tuned classical baseline.

\begin{table}[htb]
    \centering
    \caption{Performance comparison between GA baseline and CVaR-VQE variants under identical 400\,000-evaluation budget. All VQE configurations achieve 5--7\% improvement over the mutation-optimized GA, demonstrating robust quantum advantage across optimization strategies.}
    \label{tab:main_results}
    \begin{tabular}{lcccr}
        \toprule
        Method & Best Cost & Mean $\pm$ Std & $n$ & vs GA \\
        \midrule
        GA ($\mu^* = 0.22$) & 7,800 & 8,415 $\pm$ 338 & 30 & — \\
        VQE Aggressive-TPE & \textbf{6,830} & 7,778 $\pm$ 281 & 30 & \textbf{-7.6\%} \\
        VQE_noisy & 7,370 & 7,975 $\pm$ 252 & 30 & \textbf{-5.2\%} \\
        \bottomrule
    \end{tabular}
\end{table}

This result aligns with CVaR theory: since we focus on the lowest 5\% of cost 
samples, aggressive exploitation of promising parameter regions becomes more 
valuable than broad exploration.

\subsection{CVaR Risk Parameter Optimization}

The tail-focused $\alpha=0.05$ setting outperforms the broader $\alpha=0.10$ 
by 2.2\%.

Increasing the CVaR tail from 5\% to 10\% relaxes the risk focus, 
yielding degraded performance despite reduced gradient noise. This confirms that aggressive tail truncation provides the 
most benefit for our heteroscedastic quantum sampling process.

The $\alpha=0.05$ sweet spot likely reflects the balance between noise filtering 
(removing poor samples) and signal preservation (retaining sufficient statistics 
for optimization). Smaller $\alpha$ values risk over-truncation, while larger 
values dilute the risk-aware advantage. Systematic parameter space exploration 
(Appendix~\ref{sec:appendix_vqe_grid}) validates this optimal region across 
multiple circuit depths and confirms shallow circuits ($L \leq 2$) consistently 
outperform deeper alternatives.

\subsection{Circuit Depth Analysis}

Doubling circuit depth ($L=2$) provides no measurable benefit while increasing 
gate count.

Deeper circuits ($L=2$, 52~parameters) achieve statistically 
equivalent performance to shallow circuits despite additional trainable angles. 
This suggests that the 13-qubit encoding already captures sufficient entanglement 
for the routing problem structure.

From a practical perspective, shallow circuits offer clear advantages: faster 
execution, lower noise accumulation, and reduced parameter optimization complexity. 
The $L=1$ ansatz therefore represents the optimal efficiency-performance trade-off 
for near-term deployment.

\subsection{Hardware Noise Robustness}

The quantum edge survives IBM noise models with modest degradation.

Under the calibrated \texttt{ibm\_fez} noise model, quantum performance 
experiences modest degradation yet maintains statistical superiority 
over the GA~baseline ($p<0.01$). Figure~\ref{fig:noise} illustrates that while 
noise narrows the performance gap, it does not reverse the quantum-classical ranking.

\begin{figure}[h]
  \centering
  \includegraphics[width=.7\linewidth]{fig/noise_boxplot.pdf}
  \caption{Cost distributions under noiseless (white) and noisy (gray) conditions. 
           Dashed line indicates GA baseline performance.}
  \label{fig:noise}
\end{figure}

This robustness stems from CVaR's implicit noise filtering—by averaging only 
the best-performing samples, the metric naturally dampens the impact of 
noise-corrupted outliers.

\subsection{Noise-assisted Early Convergence}

While final costs remain essentially unchanged, device noise can 
accelerate convergence. Figure~\ref{fig:learning_curves_noise} overlays 
mean best–so–far trajectories for the same configuration with and without 
the calibrated \texttt{ibm\_fez} noise model (\(L=2,\,\alpha=0.05\)). The 
noisy run reaches a plateau within \textasciitilde{}\,31 objective evaluations, whereas the noiseless 
run requires \textasciitilde{}\,74 evaluations to come within 2\,\% of its final cost—a 
2.4\,\times speed-up. Appendix~Fig.~S1 quantifies this effect across individual 
trials.

\begin{figure}[h]
  \centering
  \includegraphics[width=.7\linewidth]{fig/learning_curves_noise.pdf}
  \caption{Mean best–so–far cost versus objective evaluations for the same
           CVaR-VQE configuration under noiseless (blue) and noisy (red) execution.
           Shaded bands indicate $\pm1\,\sigma$ (30 noiseless, 10 noisy trials).
           Horizontal grid lines are 100-cost increments; the bracket highlights
           the 31 vs 74 evaluation counts required to reach 102\,\% of the final
           cost.}
  \label{fig:learning_curves_noise}
\end{figure}

\subsection{Budget Scaling Analysis}

The quantum advantage is robust across evaluation budgets.

Under the fair comparison budget of 200{,}000 oracle evaluations, CVaR-VQE 
maintains a substantial 7.6\% advantage over the tuned GA baseline. This 
demonstrates that the quantum approach's superiority is not dependent on 
asymmetric evaluation budgets.

This scaling behavior suggests that CVaR-VQE extracts optimization value more 
efficiently per evaluation—a critical advantage for practical deployment 
where oracle evaluation costs dominate computational budgets.

\section{Discussion}
\label{sec:discussion}

\subsection{Coverage Shortfall and Mitigation}

Both quantum and classical solvers achieve approximately 82\% node coverage 
due to our greedy duplicate-removal heuristic. This affects both methods equally, 
preserving fair comparison. Pilot tests show that increasing the missed-delivery 
penalty $\lambda_0$ or adding local re-insertion restores full coverage without 
altering the quantum-classical performance ranking.

\subsection{Practical Deployment Considerations}

\textbf{Hardware compatibility.} Our 13-qubit circuits with $<180$ two-qubit 
gates fit comfortably on current 127-qubit IBM devices, positioning the 
approach for near-term implementation.

\textbf{Scalability pathway.} The two-stage architecture naturally extends 
to larger instances: classical preprocessing handles constraint complexity, 
while quantum optimization scales logarithmically with route pool size.

\textbf{Integration with existing systems.} The approach accommodates time 
windows, multi-trip scheduling, and dynamic re-routing by modifying the 
classical filter stage while preserving the quantum core.

\subsection{Future Research Directions}

Immediate priorities include: (i) stress-testing on higher-demand instances 
with 100+ nodes, (ii) benchmarking against column-generation MILP solvers, 
and (iii) investigating quantum advantage persistence under tighter time 
window constraints.

\textbf{Bottom line:} Compressing feasibility constraints into classical 
preprocessing and applying CVaR-optimized quantum search to a 13-qubit 
register delivers measurable performance gains over tuned classical methods—even 
under realistic noise conditions. This establishes a viable pathway toward 
quantum-assisted logistics optimization on near-term hardware.