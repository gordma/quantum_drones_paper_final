\section{Related Work}
\label{sec:related_work}

\subsection{Why direct QUBO formulations break down}
Encoding routing and VRP variants as Quadratic Unconstrained Binary
Optimisation (QUBO) explodes qubit requirements: a {\small4‑node,
6‑edge} toy instance already needs several hundred binary variables once
linearisation and penalty terms are introduced
\cite{davies_quantum_2024}.  Besides qubit count, QUBO mappings suffer
(i) delicate penalty tuning that creates large integrality gaps
\cite{kochenberger2014qubo_review} and (ii) extra two‑qubit gates to
embed logical connectivity onto hardware graphs such as Pegasus or
heavy‑hex \cite{quinton2024annealing_limits}.

\subsection{Qubit‑efficient encodings}
Log‑index schemes compress variable counts from $O(N)$ to
$O(\log N)$ qubits \cite{tan_qubit-efficient_2021,
leonidas2024qubovrp}.  Applied to VRP/VRPTW, they achieve two‑order‑of‑magnitude
reductions \cite{sundar_qubit-efficient_2024}.  The trade‑off is
deeper circuits or surrogate cost functions that can re‑introduce
approximation gaps.

\subsection{Constraint‑preserving and dual‑hybrid approaches}
Several authors eliminate penalty terms altogether via
constraint‑preserving ansätze.  Examples include Lagrangian‑dual QAOA
\cite{gabbassov2023}, QAOA++ custom mixers \cite{fuchs2022}, and
“profit‑only” post‑correction loops \cite{chang2025} for VRP variants.

\subsection{Model‑engineering and preprocessing in classical OR}
Surveys on adaptive large‑neighbourhood search (ALNS)
\cite{windras_mara_alns_2022} and modern matheuristics
\cite{boschetti_matheuristics_2024} emphasise that destroy‑and‑repair
operators, column generation and branch‑and‑price hybrids often deliver
larger gains than raw solver speed—echoing Hildebrandt’s early claim
that model implementation is the true bottleneck
\cite{hildebrandt_bottleneck_1981}.

\subsection{Black‑box hyper‑parameter optimisation}
AutoML frameworks—Hyperopt, SMAC, Optuna \cite{akiba_optuna_2019}—have
displaced manual parameter tuning in ML and OR.  In the quantum realm,
Bayesian optimisation cuts QAOA evaluations by an order of magnitude
\cite{tibaldi_bayes_qaoa_2023}, motivating our use of Optuna’s aggressive
TPE sampler in a CVaR‑VQE setting.

\subsection{Quantum samplers in hybrid pipelines}
Recent work treats NISQ devices primarily as probabilistic samplers
embedded in classical loops: Sampling‑QAOA \cite{matsuyama2025},
Generator‑Enhanced Optimisation \cite{albarran2024}, and CVaR‑QAOA
\cite{barkoutsos_improving_2020}.  Our pipeline follows this philosophy,
combining log‑index encoding with risk‑aware objectives to exploit
NISQ‑era hardware efficiently.

\subsection{Alignment with probabilistic OR and generative AI}
Cross‑entropy and generative‑model viewpoints are increasingly applied to
combinatorial optimisation \cite{caramanis2023}.  This probabilistic
stance mirrors industrial decision‑making, where operators trade
expected cost against downside risk rather than chase a single
deterministic optimum.
