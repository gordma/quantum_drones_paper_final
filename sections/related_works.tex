\section{Related Work}
\label{sec:related_work}

Most efforts to bring quantum hardware into large-scale optimisation
tackle one (or more) of three technical pain-points:
(i) cutting the sheer number of qubits needed to hold a problem
instance, (ii) enforcing hard constraints without delicate penalty
weights, and (iii) training variational circuits reliably on noisy,
depth-limited devices.  We sketch the main ideas in each line of work
and indicate how our approach builds on them.

% -------------------------------------------------
\subsection{Shrinking quantum circuit width}

\textit{Why width matters.}  
A textbook QUBO maps each binary decision to one qubit.
In routing problems that decision might be
“edge $(i,j)$ is flown at time slot~$t$ by drone~$d$,”
so even a 10-node graph quickly explodes to
$10^{4}$–$10^{5}$ variables—far beyond today’s $\mathcal{O}(10^3)$-qubit hardware
\cite{davies_quantum_2024}.

\textit{Log-index compression.}  
Tan \emph{et al.}~\cite{tan_qubit-efficient_2021} showed that if the
problem can be phrased as “pick one item from a selection of $N$
options,” the catalogue index fits in only
$\lceil\log_{2}N\rceil$ qubits.
Feasibility is verified afterwards by a classical oracle, so extra qubits
for penalty terms disappear.  Follow-up work has pushed the idea to
finantical transaction settlement problems \cite{huber_exponential_2024} and vehicle routing with time
windows \cite{leonidas_qubit_2023}, typically saving two orders
of magnitude in width.

Other encoding methods exists, like Pauli-Correlation Encoding (PCE)
of Sciorilli et al. that captures higher-body Pauli statistics and provides a polynomial reduction in qubit-resources, yet optimises a non-convex tanh surrogate and lacks any link to linear-programming or semi-definite programming relaxations \cite{sciorilli_towards_2025}. Qubit-efficient QAOA by Sundar \& Dupont \cite{sundar_qubit-efficient_2024} maps $N$ bits into $q = d + log_2 (N/d)$ qubits and shows parameter clustering on SK glasses, but still samples from depth-p circuits.

% -------------------------------------------------
\subsection{Handling constraints without fragile penalties}

\textit{The penalty problem.}  
Traditional QUBO inserts a huge coefficient into $Q$ every time a
constraint could be violated. If the coefficient is too small, illegal
solutions leak in; too large, and the quantum energy landscape develops
enormous cliffs that ruin optimisation
\cite{kochenberger_unconstrained_2014}.

\textit{Mixer-based approaches.}  
Alternating-Operator Ansätze (AOA) restrict the quantum walk to the
feasible sub-space from the outset, so \emph{no} penalty is needed
\cite{maciejewski_design_2024,fuchs_constrained_2022}.  
Custom mixers swap amplitudes only between states that already obey
all constraints.

\textit{Dual and warm-start ideas.}  
A complimentary strategy moves constraints into a smooth dual function
\cite{gabbassov_lagrangian_2025}, or seeds the quantum circuit with a
high-quality classical solution and lets the quantum layers make only
local refinements \cite{egger_warm-starting_2021}.  


% -------------------------------------------------
\subsection{Training shallow, sample-efficient optimisers}

\textit{Why shallow matters.}  Contemporary superconducting hardware can
reliably execute only $10^{2}$–$10^{3}$ two-qubit gates before decoherence
and crosstalk wash out the signal; for example, IBM’s 433-qubit
\emph{Osprey} reports average two-qubit error rates of
${\sim}2\times10^{-3}$ and recommends circuit depths below $250$
layers \cite{ibm_heavy_hex_2019,ibm_quantum_volume_2023}.  Even if those
fidelities improved, variational training still hits the
\emph{barren-plateau} barrier: cost gradients decay
exponentially with depth for generic ansätze \cite{cerezo_cost_2021,mcclean_barren_2018}.
Staying shallow—or moving to sample-based objectives such as CVaR that do
not rely on gradient information—therefore avoids both the hardware
ceiling and the theoretical wall.


\textit{Sampler-centric algorithms.}  
Sampling-QAOA, CVaR-QAOA and Generator-Enhanced Optimisation treat the
quantum computer purely as a noise-tolerant sampler, then
post-select or re-weight samples classically
\cite{matsuyama_sampling-based_2025,alcazar_enhancing_2024,barkoutsos_improving_2020}.  

\textit{Bayesian hyper-parameter search.}  
Because every circuit evaluation is expensive, Bayesian optimisation
has become the tuner of choice, cutting variational queries by an order
of magnitude in QAOA studies \cite{tibaldi_bayesian_2023}.  
Optuna’s Tree-Parzen Estimator (TPE) sampler
\cite{akiba_optuna_2019} is particularly attractive because it
integrates an early-stopping (“pruner’’) mechanism that trims off poor
parameter regions quickly.

% -------------------------------------------------
\paragraph{Our contribution in context.}
We inherit log-index compression to reduce circuit width,
further enhance through keeping feasibility logic in a classical oracle rather than penalties and additional variables,
and drive a shallow CVaR-VQE circuit with Bayesian Optimization.
The combination produces a 13-qubit solver that
beats a carefully tuned Genetic Algorithm on a realistic 30-node
drone-routing instance—demonstrating that these three strands can be
merged into a practical NISQ-era workflow.
