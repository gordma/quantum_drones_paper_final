\section{Related Work}
\label{sec:related_work}

\subsection{Why direct QUBO formulations break down}

Classical embeddings typically cast a combinational problem as a
\emph{Quadratic Unconstrained Binary Optimisation}:

\begin{equation}
\min_{x\in\{0,1\}^{n}} \; x^{\mathsf T} Q\,x,
\qquad
Q\in\mathbb{R}^{n\times n},
\label{eq:qubo_def}
\end{equation}

where every feasibility constraint is absorbed into $Q$ via large
penalty coefficients.
In a 30–40-node drone-VRP,
explicit edge–time–drone flags balloon to
$\mathcal{O}(10^{4}\!-\!10^{5})$ binary variables, implying
$\approx10^{5}$ qubits on quantum hardware.

\smallskip\noindent\textbf{Log-index remedy.}
Our work adopts the \emph{register–ancilla paradigm}
first outlined by Tan~et al.~\cite{tan_qubit-efficient_2021}:  
a $\lceil\log_2 N\rceil$-qubit \textit{index register} encodes a route
label $|r\rangle$, while a 2-qubit \textit{ancilla} stores the launch
wave $|w\rangle$.  
A projector $P_{\text{eq}}$ enforces the identity
\[
P_{\text{eq}}
\bigl(|r\rangle_{\text{reg}}\otimes|w\rangle_{\text{anc}}\bigr)
=
\delta_{X_{\text{reg}},\,X_{\text{anc}}}
\bigl(|r\rangle_{\text{reg}}\otimes|w\rangle_{\text{anc}}\bigr),
\]
so that feasibility checking is deferred to a classical oracle instead
of quadratic penalties.
The resulting 13-qubit layout replaces the dense $Q$ in
Eq.~\eqref{eq:qubo_def} by a sparse, logarithmic-width representation,
eliminating $\sim$95 \% of the qubit overhead while maintaining full
expressiveness.

\subsection{Constraint‑preserving and dual‑hybrid approaches}
Several authors eliminate penalty terms altogether via
constraint‑preserving ansätze.  Examples include Lagrangian‑dual QAOA
\cite{gabbassov_lagrangian_2025} and QAOA++ custom mixers \cite{fuchs_constrained_2022}.

\subsection{Model‑engineering and preprocessing in classical OR}
Surveys on adaptive large‑neighbourhood search (ALNS)
\cite{windras_mara_survey_2022} and modern matheuristics
\cite{boschetti_contemporary_2024} emphasise that destroy‑and‑repair
operators, column generation and branch‑and‑price hybrids often deliver
larger gains than raw solver speed—echoing Hildebrandt’s early claim
that model implementation is the true bottleneck
\cite{hildebrandt_bottleneck_1981}.

\subsection{Black‑box hyper‑parameter optimisation}
AutoML frameworks—Hyperopt, SMAC, Optuna \cite{akiba_optuna_2019}—have
displaced manual parameter tuning in ML and OR.  In the quantum realm,
Bayesian optimisation cuts QAOA evaluations by an order of magnitude
\cite{tibaldi_bayes_qaoa_2023}, motivating our use of Optuna’s aggressive
TPE sampler in a CVaR‑VQE setting.

\subsection{Quantum samplers in hybrid pipelines}
Recent work treats NISQ devices primarily as probabilistic samplers
embedded in classical loops: Sampling‑QAOA \cite{matsuyama2025},
Generator‑Enhanced Optimisation \cite{albarran2024}, and CVaR‑QAOA
\cite{barkoutsos_improving_2020}.  Our pipeline follows this philosophy,
combining log‑index encoding with risk‑aware objectives to exploit
NISQ‑era hardware efficiently.

\subsection{Alignment with probabilistic OR and generative AI}
Cross‑entropy and generative‑model viewpoints are increasingly applied to
combinatorial optimisation \cite{caramanis_optimizing_2023}.  This probabilistic
stance mirrors industrial decision‑making, where operators trade
expected cost against downside risk rather than chase a single
deterministic optimum.
