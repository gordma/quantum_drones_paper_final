% -------------------------------------------------
\section{Discussion and Outlook}
\label{sec:discussion}

\paragraph{What we have shown.}
This work provides the first evidence that a \emph{log-index, risk-aware}
VQE can out-perform a tuned Genetic Algorithm on a
battery– and wave-constrained drone VRP
\emph{within the qubit and shot budgets of today’s 127-qubit hardware}.
Our compressed 13-qubit register, millisecond oracle,
and CVaR-TPE synergy together yield a \textbf{6–8\,\%} mean-cost edge
that survives realistic \texttt{ibm\_fez} noise.

\paragraph{Limitations.}
\begin{enumerate}[nosep,leftmargin=*]
\item \textbf{Single instance.}  We benchmark on one 30–40-node graph.
      Although it is industry-derived, multi-instance validation
      (including dynamic or larger fleets) is future work.
\item \textbf{Route pool fixed \emph{a priori}.}
      The classical soft-max generator is outside the quantum loop.
      Hybrid schemes that \emph{co-evolve} the pool (e.g.\ column
      generation) remain unexplored.
\item \textbf{Baseline scope.}  GA is a strong black-box competitor,
      but other meta-heuristics (ALNS, simulated annealing, branch-and-price
      with tailored pricing) could be added to the comparison set.
\item \textbf{Noise model.}  We simulated calibrated noise rather than
      executing on hardware.  Real runs will add queue latency,
      calibration drift and SWAP overheads.
\end{enumerate}

\paragraph{Near-term extensions.}
\begin{itemize}[nosep,leftmargin=*]
\item \textbf{Adaptive $\alpha$ schedules.}  Start with a larger
      risk-averse $\alpha$ to guide exploration, then anneal toward the
      mean to refine exploitation.
\item \textbf{Layer-depth sweet spot under noise.}  Our data
      suggests $L\!=\!2$ is the “knee”; running L=3–5 on trapped-ion or
      neutral-atom devices with lower CNOT error could test whether the
      shallow optimum is hardware- or algorithm-limited.
\item \textbf{Column-generation hybrid.}  Use the quantum engine as the
      pricing oracle in a branch-and-price framework—reintroducing global
      optimality certificates without exploding qubit count.
\item \textbf{Dynamic replanning.}  Because the oracle latency is
      $\approx1$ ms, an online setting where new orders arrive mid-flight
      is within reach.  A streaming CVaR-VQE could continuously
      re-optimise the remaining routes.
\end{itemize}

\paragraph{Long-term outlook.}
As gate fidelities improve and error-mitigation becomes cheaper, the
log-index idea scales gracefully:
a 10{,}000-route pool still fits in $\,\lceil\log_2 10{,}000\rceil+2=16$
qubits, well inside the 100-qubit NISQ envelope.
The bottleneck will shift from qubit \emph{count} to circuit \emph{depth}
and oracle evaluation cost.
Our framework already exposes two independent levers—shots and Monte-Carlo
draws—to trade quantum time against classical CPU, suggesting a
resource-adaptive workflow for future heterogeneous schedulers.

\paragraph{Take-away.}
The study demonstrates that \emph{compressed, black-box quantum search}
is a viable strategy for real logistics constraints \textbf{today},
without waiting for fault-tolerance or bespoke QUBO tailoring.
We hope the open-source release stimulates follow-on work that pushes
both the industrial realism and the quantum–classical hybrid boundary.


\section{Discussion & Outlook}

In this study, we have introduced and demonstrated a novel approach to quantum combinational optimization specifically tailored to the drone vehicle routing problem (D-VRP). Our key contribution lies in shifting feasibility checking and constraint handling entirely into an efficient classical pre-processing step, allowing the quantum variational eigensolver (VQE) to act solely as a combinational "black-box learner" of optimal route subsets. Unlike traditional Quadratic Unconstrained Binary Optimization (QUBO) formulations, which quickly become qubit-intensive and prone to barren plateaus and complicated penalty-tuning requirements, our approach retains the expressive power necessary for realistic logistics problems without encountering exponential qubit overhead.

Central to our strategy is the combined use of a conditional-value-at-risk (CVaR) objective and a Bayesian hyper-parameter optimization framework (Optuna's Tree-Parzen Estimator, TPE). This combination elegantly addresses the noise challenges inherent in near-term quantum computing. Specifically, the CVaR objective explicitly focuses optimization on the worst-performing tail of the sampled cost distribution, significantly reducing the variance introduced by finite quantum shots and Monte-Carlo sampling steps. TPE then naturally exploits the smoothed optimization landscape created by CVaR, further improving the robustness and speed of convergence.

Our experiments show that this approach yields substantial practical benefits. Using just 13 qubits for realistic scenarios involving 2,048 feasible routes and 4 launch waves, our method consistently outperformed a carefully tuned Genetic Algorithm (GA) baseline by 6–8%, under identical computational budgets. Importantly, this quantum advantage remained robust even when tested against realistic hardware noise modeled after IBM's calibrated \texttt{ibm\_fez} quantum backend. Notably, shallow circuit depths (1–2 layers) were sufficient, demonstrating that such near-term quantum hardware is already capable of achieving meaningful performance improvements in industrial-scale logistics problems.

However, several open questions and limitations remain to be addressed. First, the current study used a single realistic graph instance derived from Thales UK, raising questions about generalizability to other types of graphs and constraints. Future studies should therefore systematically test performance across diverse VRP scenarios, including different geographic layouts, fleet compositions, and constraint profiles. Moreover, while our method assumes relatively straightforward cost oracle computations, its effectiveness for arbitrary black-box functions returning binary bit-string costs remains an open and exciting question.

Looking forward, our approach aligns with emerging trends in quantum optimization, moving beyond monolithic QUBO-based frameworks towards sampler-centric methodologies. Particularly promising avenues include embedding our compressed qubit encoding as a node-scorer within quantum-assisted branch-and-bound algorithms, integrating parameter-free generative quantum circuits to eliminate classical parameter optimization entirely, and exploring multi-stage CVaR objectives that explicitly model stochastic events such as weather delays or dynamic mission updates. Each of these paths leverages the compact and flexible nature of our log-index encoding and quantum-sampling strategy, underscoring the broad potential and versatility of our framework.

In conclusion, our results clearly illustrate that quantum sampler-centric optimization, leveraging classical pre-processing and minimal qubit encoding, is not only viable but can already offer tangible performance benefits for complex, constraint-rich combinational problems. We believe that our methodology provides a practical and scalable route towards near-term quantum advantage in logistics optimization and beyond.
