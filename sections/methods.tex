\section{Methods}

\subsection{Overview}

To avoid exponential qubit scaling, we split the drone routing problem into two stages. We first filter thousands of constraint-violating routes classically, then apply quantum optimization to select the best feasible subset. This hybrid approach reduces a 275-qubit QUBO problem to just 13 qubits while preserving solution quality.

\subsection{Classical Preprocessing and Feasibility Filtering}

Directly optimizing drone routes considering all constraints remains computationally challenging. Therefore, we introduce a classical preprocessing phase that dramatically reduces subsequent computational complexity. We represent the operational environment as a sparse undirected graph $G=(V,E)$, where nodes correspond to depots and customer locations. Edges between nodes are established via a Delaunay triangulation augmented by direct depot-to-customer connections, forming a practical, sparse ``star-plus-local'' topology. Each edge $(u,v)$ is annotated with travel time $t_{uv}$, battery consumption $b_{uv}$, and no-fly zone constraints $c_{uv}$.

We generate feasible drone routes via a greedy soft-max heuristic (Algorithm~\ref{alg:softmax}, Appendix~\ref{app:algorithms}). This algorithm iteratively selects subsequent nodes probabilistically based on incremental travel cost and a temperature parameter $S$ that balances exploration and exploitation. Any generated route violating battery capacity $B_{\max}$ or payload constraints $Q_{\max}$ is immediately discarded, guaranteeing feasibility by construction.

\subsection{Subset-Route Combinational Optimization}

Even after classical preprocessing, directly optimizing thousands of feasible routes remains infeasible. Thus, we recast the drone routing problem as a combinational optimization task: selecting exactly $k=10$ routes from a reduced candidate pool of $N=2048$ feasible routes, assigning each selected route to one of $M=4$ discrete launch waves. The resulting search space contains $\binom{2048}{10} \times 4^{10} \approx 3.6 \times 10^{32}$ possible configurations.

Classical heuristics manage operational constraints by removing overlapping routes within each wave using a greedy duplicate removal strategy (Algorithm~\ref{alg:dropdup}) and assigning routes to drones in ascending travel-time order (Algorithm~\ref{alg:scheduler}). This enables the quantum optimization to focus purely on combinational subset selection without additional constraint complexity.

\subsection{Qubit-Efficient Quantum Representation}

This section presents the 13-qubit encoding that enables quantum optimization on near-term devices. A logarithmic index packs 2,048 routes and four launch waves into just 13 qubits—a 95% reduction from traditional QUBO encodings. Each route-wave pair follows:

\begin{equation}
r = \left(\sum_{j=0}^{n_r-1} 2^{j} z_j\right) \bmod N, \quad 
w = \left(\sum_{j=0}^{n_w-1} 2^{j} z_{n_r+j}\right) \bmod M
\label{eq:encoding}
\end{equation}

Here, $n_r=11$ qubits encode route indices, and $n_w=2$ qubits encode wave assignments. This compressed encoding dramatically outperforms standard QUBO formulations, typically requiring hundreds of qubits for similar scenarios.

\subsection{Cost Oracle and Risk-Aware Optimization}

Robust decision-making under uncertainty requires a cost evaluation scheme sensitive to worst-case outcomes. Our cost oracle combines makespan minimization with penalty terms for operational constraint violations:

\begin{equation}
\begin{split}
C(\mathcal{S}) &= 10 T_{\max} + \lambda_E \sum_{(u,v)\in E_c} (1-\mathbb{1}_{uv})^2 + \lambda_V \sum_{v>0}(1-\mathbb{1}_v)^2 \\
&\quad + \lambda_0 \sum_{v>0}\mathbb{1}_v^{\text{miss}} + \lambda_W (\text{overbook})^2
\end{split}
\label{eq:cost}
\end{equation}

In this formulation, $\mathbb{1}_{uv}$ indicates whether an edge $(u,v)$ is traversed, and $\mathbb{1}_v$ indicates node coverage. Penalty terms strongly discourage constraint violations, ensuring robust solutions.

We minimize the Conditional Value-at-Risk (CVaR) to handle the inherent stochasticity in quantum sampling:

\begin{equation}
J_{\alpha}(\mathbf{c}) = \frac{1}{\alpha K}\sum_{i=1}^{\alpha K} c_{(i)}
\label{eq:cvar}
\end{equation}

This tail-average focuses the optimizer on worst-case performance, filtering out high-variance samples that could mislead gradient-free search.

\subsection{Bayesian Hyperparameter Optimization}

The inherently noisy and non-differentiable quantum optimization landscape necessitates specialized optimization strategies. We employ Optuna's Tree-Parzen Estimator (TPE), which explicitly models performance distributions by fitting separate density estimates for high-performing and low-performing regions. New trial parameters are chosen to maximize expected improvement, efficiently handling heteroscedastic and non-convex cost surfaces. Two TPE configurations---Default and Aggressive---are systematically evaluated, with median-based early stopping criteria to manage computational costs.

\subsection{Classical Benchmarking}

Rigorous evaluation of quantum solutions necessitates benchmarking against well-established classical optimization methods. We adopt a Genetic Algorithm baseline using bit-string encoding, roulette-wheel selection with elitism, adaptive mutation probabilities (high: 0.06, low: 0.012), and two-point crossover. Computational budgets are strictly matched between classical and quantum methods to ensure fair comparisons, with comprehensive tuning results documented in Appendix~\ref{app:algorithms}.

\subsection{Experimental Configuration}

Table~\ref{tab:hyperparams} provides a comprehensive overview of the hyperparameters and configurations used across experiments. Quantum optimization employs a hardware-efficient variational ansatz consisting of $L \in \{1,2\}$ layers, with $N_{\text{shots}}=4096$ circuit evaluations and $N_{\text{mc}}=2000$ Monte Carlo samples for robust cost estimation.

This structured methodological framework systematically reduces quantum optimization complexity while maintaining robust solution quality, demonstrating a viable path toward practical quantum advantage in realistic drone logistics applications.


\subsection{Default Hyper-parameters}

\begin{table}[H]
\centering
\begin{tabular}{lcc}
\toprule
Parameter & Symbol & Value \\
\midrule
Battery capacity & $B_{\max}$ & 80 \\
Payload capacity & $Q_{\max}$ & 50 \\
Selected routes & $k$ & 10 \\
Circuit shots & $N_{\text{shots}}$ & 4\,096 \\
MC samples & $N_{\text{mc}}$ & 2\,000 \\
CVaR tail & $\alpha$ & 0.05 \\
Ansatz layers & $L$ & 1–2 \\
Oracle budget / trial &  & $10^{5}$ \\
\bottomrule
\end{tabular}
\caption{Default hyper-parameters shared across all experiments.}
\end{table}