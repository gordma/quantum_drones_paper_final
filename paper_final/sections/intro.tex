\section{Introduction}

\subsection{Drone logistics at an engineering inflection point}

Recent large-scale pilot programmes by UPS\,Flight Forward, Zipline, Wing and others demonstrate that unmanned aerial vehicles (UAVs) can deliver small parcels in minutes while lowering both labour and \text{CO\textsubscript{2}} footprints.\cite{murray2020survey}  
Yet each real-world deployment must respect \emph{simultaneous} constraints: limited battery energy, strict payload caps, no-fly corridors, and air-traffic ``wave’’ launch schedules mandated by regulators.  
In principle every such rule can be embedded in a mixed-integer programming (MILP) or branch-and-price formulation, but industrial case studies consistently report that \emph{human modelling effort---not CPU time---dominates project cost}.  
Bangert calls data cleansing and modelling ``the steps that require \emph{most} of the human time and effort’’ in optimisation projects\cite{bangert2012industrial}, while hybrid CP/column-generation surveys list bespoke pricing routines as a \emph{major barrier}\cite{rossi2006hcp}.  
Even powerful commercial MILP solvers such as Gurobi or CPLEX only shine after weeks of expert model sculpting and penalty tuning.
Consequently, organisations favour fast heuristic tooling even when exact methods could yield better objective values.

\subsection{From feasibility to combinatorial choice: the Subset-Route D-VRP}

We adopt a middle-ground abstraction tailored to NISQ hardware.  
First, a greedy soft-max generator creates a pool of $N=2\,048$ \emph{feasible} routes on a 30--40-node tactical graph, fully respecting battery, payload and no-fly constraints.  
The operational decision therefore collapses to selecting exactly $k = 10$ route--wave pairs\footnote{Four launch waves $\Rightarrow \lceil\log_2 4\rceil = 2$ qubits for wave index.} out of this pool.  
The search space contains ${2\,048 \choose 10}\times4^{10}\!\approx 2.6\times10^{26}$ possibilities---far beyond exhaustive enumeration but representable with only
%
\[
\lceil\log_2 N\rceil + \lceil\log_2 4\rceil = 11 + 2 = 13
\]
%
qubits via index encoding.  All feasibility reasoning remains in classical code; quantum resources are reserved for the \emph{combinatorial} challenge.

\subsection{Why classical baselines plateau}

A well-tuned Genetic Algorithm (GA) is the de-facto black-box baseline: it requires only the same cost oracle we already possess for feasibility checking.  
However, empirical GA improvement typically stalls after $\mathcal{O}(10^{5})$ oracle calls on noisy surfaces, and the algorithm is \emph{risk-neutral}: it optimises the mean of a heteroscedastic objective without regard to tail costs.

\subsection{Quantum as a stochastic search engine: gap in the literature}

A growing body of work uses quantum hardware to \emph{sample} from rich distributions---reservoir computing, qGANs, probabilistic QAOA---rather than to collapse onto a single ``winner’’ bit-string.\cite{schuld2019quantum}  
Yet existing quantum VRP studies still target toy instances ($\le 50$ nodes), ignore battery/wave constraints, and consume 100--500 qubits for direct QUBO encodings.\cite{vogel2022quantumvrp}  
Whether NISQ devices can \emph{outperform strong classical baselines on realistic, constraint-heavy drone routing} therefore remains open.

\subsection{Our compressed, risk-aware pipeline}

We answer that question with a systematic approach:

\begin{enumerate}[leftmargin=*,nosep]
\item \textbf{Compressed log-index encoding} reduces the decision register to 13 qubits for the full 2\,048\,$\times$\,4 design space.
\item \textbf{Millisecond black-box oracle} evaluates any candidate set in $\approx1$\,ms, inheriting exactly the same engineering cost as the GA baseline.
\item \textbf{CVaR objective ($\alpha\!\approx\!0.05$)} focuses optimisation on the worst-cost tail, naturally down-weighting bit-strings that appear with vanishing shot probability.
      Heteroscedastic noise stems from (i) quantum shots (binomial variance $p(1-p)$) and (ii) Monte-Carlo resampling of $2\,000$ route subsets.
      CVaR truncates both noise sources.
\item \textbf{Optuna TPE Bayesian search} exploits the smoother, risk-focused landscape without assuming homoscedastic Gaussian noise, yielding a synergy we do not observe with na\"ive gradient-free methods.
\end{enumerate}

\subsection{Headline results}

On the 30--40-node instance our $1$--$2$-layer CVaR-VQE achieves a \textbf{6--8\,\% lower mean cost} than a mutation-swept GA under an identical budget of $10^{5}$ oracle evaluations, and the edge persists under a calibrated \texttt{ibm\_fez} noise model.  
The entire pipeline is released under the MIT licence and comprises $<\!1\,000$ lines of Python.

\subsection{Contributions}

\begin{enumerate}[leftmargin=*,nosep]
\item First risk-aware quantum optimiser to surpass a tuned GA on a realistic battery- and wave-constrained D-VRP with only 13 qubits.
\item Demonstrates how shifting feasibility to a classical generator plus log-index encoding slashes qubit requirements without manual MILP modelling.
\item Provides an open, engineering-light toolkit that invites hybrid exact extensions (e.g.\ column generation) for future work.
\end{enumerate}

\subsection{Roadmap}

Section~\ref{sec:problem} formalises the Subset-Route D-VRP and the black-box oracle.  
Section~\ref{sec:methods} details the compressed-qubit VQE, CVaR objective and Bayesian optimiser.  
Section~\ref{sec:experiments} summarises experimental settings; Section~\ref{sec:results} presents results and ablation studies.  
We conclude with limitations and future directions in Section~\ref{sec:discussion}.
